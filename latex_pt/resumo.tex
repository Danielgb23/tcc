%\fronttitle{Resumo}

%Resumo redigido obrigatoriamente em português, contendo no máximo 500 palavras.

%\cleardoublepage

\fronttitle{Resumo}
%Atom interferometry is a technique that allows us to measure accelerations and rotations, for example, like gravity, through the quantum states of the atoms in the experiment by exciting and controlling these Rubidium atoms with lasers and other electromagnetic fields. In our experiment, we used it to measure gravity with remarkable precision as well as form a Bose-Einstein condensate with the atoms to implement a new quantum technique called Delta Kick Squeezing (DKS) that improves the precision of the experiment even further.
A interferometria atômica é uma técnica que nos permite medir acelerações e rotações, como, por exemplo, a gravidade, por meio dos estados quânticos dos átomos no experimento, excitando e controlando esses átomos de Rubídio com lasers e outros campos eletromagnéticos. Em nosso experimento, usamos essa técnica para medir a gravidade com precisão notável, assim como para formar um condensado de Bose-Einstein com os átomos, a fim de implementar uma nova técnica quântica chamada ``Delta Kick Squeezing'' (DKS)

%This document describes improvements made to the Raman laser system during an internship at SYRTE in the Observatory of Paris. The Raman lasers are responsible for coupling the atoms to the excited state, from whose measurements we can infer the acceleration of gravity. The laboratory already had an atom interferometry gravimeter installed that was being modified for the implementation of the DKS technique. The improvements made include the setup of a tapered amplifier to increase the power of the existing Raman lasers, an offset lock, locking the Raman controller laser (Raman 2) to the repump laser so that the two photon detuning is stabilized, and also a power stabilization for the Raman lasers by locking the tapered amplifier's current.
Este documento descreve as melhorias feitas no sistema de laser Raman durante um estágio no SYRTE, no Observatório de Paris. Os lasers Raman são responsáveis por acoplar os átomos ao estado excitado, a partir de cujas medições podemos inferir a aceleração da gravidade. O laboratório já possuía um gravímetro de interferometria atômica instalado que estava sendo modificado para a implementação da técnica DKS. As melhorias realizadas incluem a configuração de um amplificador cônico para aumentar a potência dos lasers Raman existentes, um bloqueio de desvio de frequência, bloqueando o laser controlador Raman (Raman 2) ao laser de repump para que o desvio de dois fótons seja estabilizado, e também uma estabilização de potência para os lasers Raman, bloqueando a corrente do amplificador cônico.

%\textbf{Internship description}

%\textbf{20 pages max total}

%\url{https://intranet.espci.fr/enseignement/stages/projets-de-recherche-3a}

%\url{https://theses.hal.science/tel-00070861}

%Our team at SYRTE develops inertial sensors (gyrometers, accelerometers...) based on atom interferometry technics. The development of these instruments benefits from the maturity of ultracold atom technology and the advantage of light beamsplitters, easy to implement and efficient, namely two photon transitions and more specifically stimulated Raman transitions. If these methods allow now for the development of commercial products with applications in geophysics on the field, and of onboard instruments in ships or planes for inertial navigation and geoscience, increasing significantly the performances of such instruments remains possible, in particular by using advanced quantum metrology methods to surpass the standard quantum limit. 

%The aim of this intership is the implementation of the "Delta-Kick squeezing" (DKS) technique, recently proposed by Robin Corgier, currently a postdoctoral fellow at SYRTE, and his collaborators. This DKS rely on the engineering of atom atom interactions in a BEC in free fall. Such interactions induce strong correlations between the atoms, and lead to squeezing in the population difference between the two interferometer paths, and eventually to phase sensitivity below the standard quantum limit. 

%The intern will work on implementing this method in a free-falling atom interferometer, based on the use of Raman light beamsplitters and ultra-cold atoms produced by evaporative cooling. The work, essentially experimental in nature, will first consist in optimizing the preparation sequence of ultra-cold atoms, to obtain Bose Einstein condensates in a robust and efficient way, and in optimizing the detection method of the two output ports of the interferometer. The intern will then demonstrate the possibility of realizing strongly spin- squeezed states through atomic lensing methods based on pulse sequences realized with highly detuned high power laser beams. Finally, he or she will study the impact of the use of these quantum states in an interferometer on the sensitivity of measurements. He or she will conduct the experimental studies and participate in the analysis of the results. He or she will have extensive theoretical support for the modeling of the experiment, the optimization of the measurement sequence and for the analysis of the results.


%Le rapport doit présenter la question posée, l’état de l’art, le raisonnementscientifique, les moyens mis en œuvre, les résultats, l’interprétation, la discussion et la conclusion, de manière dense et claire. Il est souvent nécessaire de réordonner le déroulement chronologique du travail et de sélectionner votre contenu, de manière à présenter non pas une liste de tâches successives mais une véritable logique scientifique. Le rapport écrit doit ainsi se rapprocher autant que ce peut de la densité et la rigueur d’un article scientifique (importance du sujet, nouveauté de la question, de l’approche, raisonnement, critère statistiques. . . ), et non être un simple rapport de tâches effectuées. Le rapport n’est pas un cahier de laboratoire.
