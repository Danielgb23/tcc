0 In this presentation I will give an outline of the experiment itself and then explain some of my work on the Raman laser system.

1 The states used on the interferometry are the two hyperfine levels on the Ground state of the Rubidium atom. To make the transition we could use a microwave signal. But then the momentum transfer would be too low and our inteferometer would be too short. So we use a two photon transition back and forth from a higher state called a Raman transition with lasers detuned from each other with the level's difference in energy. These lasers are around 780nm.

2 To help understand the atom interferometry I will explain other related experiments. The mach zender light interferometer is made up of two beam splitters and two mirrors to separate and combine again the two arms of the interferometer. We will see an amount of interference depending on the difference of phase between the two arms.

3 In the Ramsey interferometry we do a pi/2 pulse that will send a two level quantum system to the equator of the bloch sphere. Where if there's a detuning of the lasers for the pulses the quantum state will do a precession around the equator. Then we do another pi/2 pulse and depending on the interrogation time and the amount of detuning we will get a different probability of getting the excited state in the end by rotation the state around the y axis again. Here you can see the fringes if we scan the detuning of the laser. The longer the interrogation time the finer are the fringes as we could infer from this formula.

4 Now finally we talk about the atom interferometry. As you know matter can be thought as a wave too in quantum mechanics so we do something very similar to the mach zender interferometer where a pi/2 pulse is like the first beam splitter, splitting the atom matter wave in two, excited with a momentum and not. Then we do a pi pulse that is equivalent to the mirrors and that changes the arms from diverging to converging so we can combine then. And finally another pi/2 pulse that is the other beam splitter and conbines the arms so they can interfere. The space covered by the arm with momentum will make a difference in phase. And this space corresponds to the fall trajectory of the atoms.

5 When the atoms fall and we do a higher and higher interrogation time, the fringes of the interferometer will look like this chirp because the atoms accelerate. A chirp is a signal that changes in frequency with time. To get the value of gravity from the interferometer we chirp the detuning of the two raman lasers. The chirp induces another phase difference. This is like changing the detuning of the Ramsey interferometer with time. With more interrogation time we also get finer fringes.

6 Then we scan the chirp rate. At a specific chirp rate we get a resonance with gravity and we will always get perfect interference. So we get a valley at this spot for all interrogation times. From this chirp rate we can infer gravity.

7 I will not go into a lot of detail for this. In our experiment we are limited in precision by the uncertainty of the inequality of the uncertainty principle. But we can cheat it a bit by doing squeezing. Where we trade uncertainty in one quantum variable for uncertainty in another one. The delta kick squeezing to be implemented in our experiment does that by creating a bose eistein condensate to do interactions between the atoms which create the squeezing.

8 Now I am going to talk about my work. We needed more power on the raman lasers. So to do that without changing a lot the experiment setup we used a Tapered amplifier or TA. The TA is a gain medium like you have in a laser but it does not have a cavity. So you make a laser you have pass through it and it comes amplified.

9 Though the semiconductor chip where there's the gain medium is flat. Which makes the output laser is astigmatic. This means it has a different focus in the vertical and horizontal direction. So we need cyllidrical lenses to correct this. In my setup I used one spherical lens and another cyllidrical one to correct the astigmatism.

10 The TA can have it's behavior altered negatively by a back reflection from the rest of the system. This means we need an optical isolator which works like a diode but for polarized light. In the foward direction a magnetic field rotates the polarized light so it matches the polarizer on the output. But in the other direction the light rotates in the wrong direction and is blocked b the polarizer on the input.

11 Then we have to put our laser into a fiber. This is just done by a lens that focuses the beam into the fiber's entrance. Which is a very tight spot.

12 Here is the build for the TA amplification. We combine the two Raman lasers using a beam splitter. Then we go through a telescope to correct the collimation. We use a waveplate that rotates the polarization to get the right polarization to enter the TA and we go through the TA which has a lens on it's input. We correct the astigmatism and collimate the beam with a spherical and cyllidrical lenses. Go through the isolator and go through another telescope to make the size of the beam match the size needed for the fiber. Then we couple the beam into the fiber. There's also a wacveplate and polorized beam splitter to clean the polarization of the beam.

13 We want to lock the Raman laser system to the repump laser to get a stable frequency. So we combine the two lasers into a photodiode which will give a signal corresponding to the beatnote of the two lasers which is caused by their difference in frequency. We convert the freauency of this beatnote into a voltage which is subtracted to a reference and transformed into an error signal. Through integrators we control the current and piezo of the R2 locking it's frequency into place.

14 This is similar to the offset lock. But instead of locking the frequency we lock the power of the lasers through the TA's current which changes the amount of amplification. We get the power generated through a photodiode we also integrate it and then use this error signal to modulate the current of the TA. You can see that the power measured of the laser gets a much lower variance when it's locked.

To conclude. With this internship I was able to learn a lot in multiple fields related to experimental quantum physics