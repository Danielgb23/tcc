\section{Conclusão}

O trabalho no sistema de laser do experimento, juntamente com a participação em algumas partes do próprio experimento, foi uma experiência que desenvolveu muitas novas habilidades e conhecimentos, assim como a base para colocá-los em prática. Isso inclui óptica, física atômica, eletrônica, mecânica quântica e engenharia de controle, entre outros. Na minha contribuição para o experimento, com o \gls{TA} conseguimos obter uma potência significativamente maior nos feixes Raman sem muitas modificações no arranjo experimental. Além disso, o trava de offset e a estabilização de potência também não ocuparam muito espaço nem exigiram muitas mudanças, e a estabilização de potência trouxe melhorias muito significativas para a estabilidade. Ajustando o desacordo Raman e o corrente do TA, conseguimos atingir a frequência de acoplamento/Rabi esperada.

Este relatório também é uma boa documentação que descreve as modificações implementadas e os desafios encontrados nesse processo, que agora podem ser superados mais facilmente com as informações fornecidas. O estagiário também participou da produção de documentação nos cadernos de laboratório do experimento, destacando a importância de registrar todas as modificações executadas de forma clara para que outra pessoa possa continuar o trabalho facilmente no futuro.
